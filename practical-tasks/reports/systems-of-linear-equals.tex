\documentclass[a4paper, 12pt]{article}
\usepackage[english, russian]{babel}
\usepackage[T2A]{fontenc}
\usepackage[utf8]{inputenc}
\usepackage{graphicx}
\usepackage{stmaryrd}
\usepackage{amsmath}
\usepackage{amssymb}
\usepackage{amsfonts}
\usepackage{array}
\usepackage{booktabs}
\usepackage{wrapfig}
\usepackage{caption} 
\usepackage{listings}

\begin{document}
    \begin{center}
        \subsubsection*{Отчёт по практическим заданиям}
        \subsubsection*{Прикладная математика. Лекция №3.}
    \end{center}

    \quad \textbf{Задача 1.} Найти решение системы линейных уравнений с помощью методов Жордана, Гаусса:
    
    \begin{center}

        1) $A = \left(
            \begin{array}{ccc}
                5.64 & -4.52 & 4.57 \\
                -2.17 & 1.36 & -5.53 \\
                8.77 & -2.78 & 5.44 \\
            \end{array}
         \right),$ $B = \left(
            \begin{array}{c}
                8.32 \\
                7.21 \\
                7.56 \\
            \end{array}
         \right);\newline$

         2) $A = \left(
            \begin{array}{cccc}
                2.34 & -1.84 & 0.32 & 0.11 \\
                -1.19 & 0.43 & -0.52 & 3.37 \\
                0.33 & 0.61 & 7.75 & -2.18 \\
                -1.53 & 0.81 & 0.94 & -4.82 \\
            \end{array}
         \right),$ $B = \left(
            \begin{array}{c}
                2.22 \\
                -5.26 \\
                0.15 \\
                -3.74 \\
            \end{array}
         \right);$

    \end{center}

    \quad \textbf{Решение.} (/practical-task/systems-of-linear-equals/task-1.m) Суть решения СЛАУ методом Гаусса сводится к приведению матрицы системы к правотреугольному виду (прямой ход решения), в т. ч. к виду:
    
    \begin{center}
        $ A = \left(
            \begin{array}{cccc}
                a_{11} & a_{12} & ... & a_{1n} \\
                0 & a_{22} & ... & a_{2n} \\
                ... & ... & ... & ... \\
                0 & 0 & ... & a_{nn} \\    
            \end{array}
        \right)$
    \end{center}

    \quad Тогда опредедение корней СЛАУ (обратный ход) происходит по формуле: 

    \begin{center}
        $x_i = \dfrac{1}{a_{i i}} (b_i - \displaystyle\sum\limits_{k=i}^{n} a_{ik} x_k) $
    \end{center}

    \quad Метод Жордана отличается от метода Гаусса в том, что матрица приводится к диогональному виду. 
    Надо отметить что это возможно только если матрица системы не вырождена. После приведения матрицы к диагональному виду алгоритм нахождения корней тривиален.  

    \quad Таким образом вектор решений первой системы $\vec{X} = (-2.5333, -2.2592, 1.7173)$, а второй $\vec{X} = (-0.8723, -0.2924, 6.4317, 6.8038)$ .

    \quad \textbf{Задача 2.} Решить систему линейный уравнений используя метод Холецкого-Краута:

    \begin{center}
        $A = \left(
            \begin{array}{cccc}
                3.12 & 1.44 & 0.72 & -0.34 \\
                1.44 & 2.67 & 0.35 & -0.78 \\
                0.72 & 0.35 & 7.15 & 0.25 \\
                -0.34 & -0.78 & 0.25 & 6.08 \\
            \end{array}
        \right)$, $B = \left(
            \begin{array}{ccc}
                3.05 \\
                2.75 \\ 
                0.86 \\
                2.53 \\
            \end{array}
        \right)$
    \end{center}

    \quad \textbf{Решение.} (/practical-task/systems-of-linear-equals/task-2.m) Сначала решим систему общим методом Краута (частным случаем которого является схема Холецкого).
    Для этого необходимо представить матрицу системы в виде произведения двух треугольных матриц, т. е. в виде $ A = LC$, тогда система примет вид:

    \begin{center}
        $AX = B \implies
        \begin{cases}
            LY = B \\
            CX = Y \\
        \end{cases} (1)$
    \end{center}

    \quad Таким образом задача сводится к решению двух систем с треугольными матрицами, где для поиска решения нужно повторить обратный ход из метода Жордана-Гаусса дважды. 

    \begin{center}
        $\begin{cases}
            y_i = b_i - \displaystyle\sum\limits_{k=1}^{i-1} b_k l_{i k} \\
            x_i = \dfrac{1}{c_{ii}} (y_i - \displaystyle\sum\limits_{k=i}^{n} c_{i k} x_{k})
        \end{cases}$
    \end{center}

    \quad Cхема Холецкого как частный случай метода Краута предполагает разложение матрицы системы в виде $A = L L^T$, далее ход решения аналогичен.
    Надо также уточнить, что схема холецкого работает только для симметричных матриц ($A = A^T$).

    \quad Вектор решений системы $\vec{X} = (-0.0061628,   0.561,  0.65072,   0.84371)$, что полностью сходится с ответом. 

    \quad \textbf{Задача 3.} Решить систему линейных уравнений методом прогонки.

    \begin{center}
        $A = \left(
            \begin{array}{ccccc}
                3.15 & 2.12 & 0 & 0 & 0 \\
                0.23 & 4.72 & 1.62 & 0 & 0 \\
                0 & 1.91 & 8.31 & 2.19 & 0 \\
                0 & 0 & 0.79 & 3.21 & 1.94 \\
                0 & 0 & 0 & 0.29 & 5.16 \\ 
            \end{array}
        \right)$, $B = \left(
            \begin{array}{c}
                2.31 \\
                3.88 \\ 
                5.06 \\ 
                4.19 \\ 
                3.67 \\
            \end{array}
        \right)$
    \end{center}

    \quad \textbf{Решение.} (/practical-task/systems-of-linear-equals/task-3.m) Данный метода устойчив только для систем с трёхдиагональными матрицами. В сущности он представляет собой нахождение коэфицентов в уравнении 
    
    \centering $x_i = \alpha_i x_{i+1} + \beta_i$

    \quad Это ураанении в свою очередь связывает корни системы между собой, тем самым задавая рекурентую последовательность корней. Вектора решений данной системы $\vec{X} = (0.2152,   0.2376,   0.6634,   0.7366,   0.8514)$

\end{document}
